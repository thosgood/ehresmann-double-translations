\documentclass{article}

\usepackage[margin=1.6in]{geometry}

\title{Double categories and structured categories}
\author{Charles Ehresmann}
\date{4\textsuperscript{th} of February, 1963}

\newcommand{\origcit}{%
  \textsc{Ehresmann, Charles.}
  ``Catégories doubles et catégories structurées''.
  \emph{C. R. Acad. Sc.} \textbf{256} (1963), 1198--1201.
  Presented by Jean Leray.
}


%% Standards %%
\usepackage{amssymb}
\usepackage{amsmath}
\usepackage{hyperref}
\usepackage{xcolor}
\hypersetup{colorlinks,linkcolor={blue!50!black},citecolor={blue!50!black},urlcolor={blue!80!black}}
\usepackage{enumerate}
\usepackage{graphicx}
\usepackage{footmisc}


%% Typography %%

\usepackage{fouriernc}
\usepackage{Baskervaldx}


%% Theorem environments %%

\usepackage{amsthm}

\newenvironment{itenv}[1]
  {\phantomsection\par\medskip\noindent\textbf{#1.}\itshape}
  {\par\medskip}

\newenvironment{rmenv}[1]
  {\phantomsection\par\medskip\noindent\textbf{#1.}\rmfamily}
  {\par\medskip}


%% Shortcuts %%

\renewcommand{\geq}{\geqslant}
\renewcommand{\leq}{\leqslant}

\newcommand{\oldpage}[1]{\marginpar{\footnotesize$\Big\vert$ \textit{p.~#1}}}
\newcommand{\todo}{{\color{purple}\textbf{TO-DO }}}
\newcommand{\unsure}[1]{{\color{purple}\textbf{#1}}}

\newcommand{\CC}{\mathcal{C}}
\newcommand{\dotc}{{\mathbin{\bullet}}}
\newcommand{\botc}{{\mathbin{\bot}}}
\DeclareMathOperator{\sq}{\square}
\DeclareMathOperator{\vsq}{{\rotatebox{90}{$\boxminus$}}}
\DeclareMathOperator{\hsq}{\boxminus}
\newcommand{\vmult}{{\,\rotatebox{90}{$\boxminus$}\,}}
\newcommand{\hmult}{{\,\boxminus\,}}
\DeclareMathOperator{\FF}{\mathcal{F}}
\DeclareMathOperator{\NN}{\mathfrak{N}}
\newcommand{\MM}{\mathfrak{M}}
\newcommand{\KK}{\mathcal{K}}
\renewcommand{\SS}{\mathcal{S}}
\newcommand{\TT}{\mathcal{T}}


%% Document %%

\begin{document}

\maketitle

\hrule
\begin{itenv}{Note from the translator}
This document is a translation from French of the article

\medskip
{\normalfont\origcit}

\medskip
produced with permission from \todo
\end{itenv}
\hrule

\begin{abstract}
  Definition of structured categories; the particular case of double categories, which admit a category of squares as a quotient category.
\end{abstract}

%% Content %%

\todo \unsure{fix $\vmult$ and $\hmult$; fix $\dotc$}

\section{Double categories}
\oldpage{1}

\begin{rmenv}{Definition}
  We define a \emph{double category} to be a class $\CC$ endowed with two composition laws, denoted $\dotc$ and $\botc$, satisfying the following conditions:

  \begin{enumerate}
    \item $(\CC,\dotc)$ is a category, denoted $\CC^\dotc$;
      the right and left \todo of $f\in\CC$ will be denoted by $\alpha^\dotc(f)$ and $\beta^\dotc(f)$ respectively, and the class of \todo by $\CC_0^\dotc$;
    \item $(\CC,\botc)$ is a category, denoted $\CC^\botc$;
      the \todo of $f\in\CC^\botc$ will be denoted by $\alpha^\botc(f)$ and $\beta^\botc(f)$ respectively, and the class of \todo by $\CC_0^\botc$;
    \item The maps $\alpha^\dotc$ and $\beta^\dotc$ (resp. $\alpha^\botc$ and $\beta^\botc$) are functors from $\CC^\botc$ to $\CC^\botc$ (resp. from $\CC^\dotc$ to $\CC^\dotc$);
    \item \emph{Axiom of permutability.}
      If the composites $k\dotc h$, $g\dotc f$, $k\botc g$, and $h\botc f$ are defined, then
      \[
        (k\dotc h)\botc(g\dotc f)
        = (k\botc g)\dotc(h\botc f).
      \]
  \end{enumerate}

  Let $\CC$ be a class endowed with two composition laws $\dotc$ and $\botc$ satisfying axioms 1 and 2; consider the following axioms:

  \begin{enumerate}
    \item[3\textquotesingle.]
      $\CC_0^\dotc$ (resp. $\CC_0^\botc$) is stable with respect to $\botc$ (resp. to $\dotc$);
    \item[4\textquotesingle.]
      If the composites $k\dotc h$, $g\dotc f$, $k\botc g$, and $h\botc f$ are defined, then both $(k\dotc h)\botc(g\dotc f)$ and $(k\botc g)\dotc(h\botc f)$ are defined and are equal to one another.
    \item[5.]
      For all $f\in\CC$, we have
      \[
        \begin{aligned}
          \alpha^\dotc(\alpha^\botc(f))
          = \alpha^\botc(\alpha^\dotc(f)),
          &\qquad
          \beta^\dotc(\beta^\botc(f))
          = \beta^\botc(\beta^\dotc(f));
        \\\alpha^\dotc(\beta^\botc(f))
          = \beta^\botc(\alpha^\dotc(f)),
          &\qquad
          \alpha^\botc(\beta^\dotc(f))
          = \beta^\dotc(\alpha^\botc(f)).
        \end{aligned}
      \]
  \end{enumerate}
\end{rmenv}

\begin{itenv}{Proposition}
  For $(\CC,\dotc,\botc)$ to be a double category, it is necessary and sufficient that conditions 1, 2, 3\textquotesingle, 4\textquotesingle, and 5 be satisfied.
  In this case, $\CC_0^\dotc$ (resp. $\CC_0^\botc$) is a subcategory of $\CC^\dotc$ (resp. $\CC^\botc$).
\end{itenv}

A \emph{double subcategory} of a double category $\CC$ is a subclass $\CC'$ of $\CC$ that is a subcategory of $\CC^\dotc$ and of $\CC^\botc$;
then $\CC'$ is a double category for the composition laws induced by $\dotc$ and $\botc$.

\begin{rmenv}{Definition}
  Let $\CC$ be a double category;
  we define a \emph{left ideal} (resp. \emph{right ideal}) of $\CC^\botc$ to be a subcategory $I^\botc$ of $\CC^\botc$ such that $\CC\dotc I^\botc$ (resp. $I^\botc\dotc\CC=I^\botc$), where $\CC\dotc I^\botc$ (resp. $I^\botc\dotc\CC$) is the class of composites $f\dotc g$ (resp. $g\dotc f$) for $g\in I^\botc$ and $f\in\CC$.
  We similarly define an \emph{ideal} of $\CC^\dotc$.
\end{rmenv}

\oldpage{2}

\begin{itenv}{Proposition}
  Let $\CC$ be a double category;
  a left ideal $I^\botc$ of $\CC^\botc$ is a \todo\footnote{\label{fn1}\emph{Espèces de structures locales; élargissements de catégories}, Séminaire Top. et Géo Diff. (Ehresmann), \textbf{III}, Paris, 1961; Jahres. Deutsch. Math. Ver., \textbf{60}, 1957, p.~49.} over $\CC^\dotc$ for the composition law $(f,g)\mapsto f\dotc g$ if and only if $f\dotc g$ is defined, where $f\in\CC$ and $g\in I^\botc$.
  The corresponding category $\mathcal{E}(I^\botc)$ \unsure{is this right?} of hypermorphisms\footref{fn1} is a double category for the composition laws \unsure{is this right?}
  \[
    (f',g')\dotc(f,g)
    = (f'\dotc f,g)
  \]
  if and only if $g'=f\dotc g$; further \unsure{is this right? or is it two joined iffs}
  \[
    (f',g')\botc(f,g)
    = (f'\botc f,g'\botc g)
  \]
  if and only if $f'\botc f$ and $g'\botc g$ are defined.
\end{itenv}


\section{Double categories of squares}

Let $\CC_1$ and $\CC_2$ be two categories with the same class of \todo.
Let $\sq(\CC_2,\CC_1)$ be the set of quadruples $(g_2,g_1,f_1,f_2)$, with $f_i,g_i\in\CC_i$ for $i=1,2$, such that
\[
  \begin{aligned}
    \alpha(f_1)
    = \alpha(f_2),
    &\qquad
    \alpha(g_1)
    = \beta(f_2);
  \\\beta(f_1)
    = \alpha(g_2),
    &\qquad
    \beta(g_1)
    = \beta(g_2).
  \end{aligned}
\]
We define two composition laws on $\sq(\CC_2,\CC_1)$:
\begin{itemize}
  \item \emph{Longitudinal} multiplication
    \[
      (g'_2,g'_1,f'_1,f'_2)\vmult(g_2,g_1,f_1,f_2)
      = (g'_2,g'_1g_1,f'_1f_1,f_2)
    \]
    if and only if $f'_2=g_2$ \unsure{???};
  \item \emph{Lateral} multiplication
    \[
      (g'_2,g'_1,f'_1,f'_2)\hmult(g_2,g_1,f_1,f_2)
      = (g'_2g_2,g'_1,f_1,f'_2f_2)
    \]
    if and only if $f'_1=g_1$ \unsure{???}.
\end{itemize}

\begin{itenv}{Proposition}
  $\sq(\CC_2,\CC_1)$ is a double category for longitudinal and lateral multiplication.
\end{itenv}

Suppose that $\CC=\CC_1=\CC_2$;
recall\footref{fn1} that a \emph{square} in $\CC$ is an element $(g_2,g_1,f_1,f_2)\in\sq(\CC,\CC)$ such that $g_1f_2=g_2f_1$.

\begin{itenv}{Corollary}
  The class $\sq\CC$ of squares in $\CC$ is a double subcategory of $\sq(\CC,\CC)$.
\end{itenv}

\begin{itenv}{Corollary}
  Let $\CC$ be a double category;
  then $\CC^\dotc$ admits the longitudinal category $\vsq(\CC_0^\dotc,\CC_0^\botc)$ as a quotient category\footref{fn1}, where $\CC_0^\dotc$ (resp. $\CC_0^\botc$) is endowed with its structure as a subcategory of $\CC^\botc$ (resp. of $\CC^\dotc$).
\end{itenv}


\section{Functors into a double category}

Let $\Gamma$ be a category and $\CC$ a double category;
let $\FF(\CC^\dotc,\Gamma)$ be the class of functors from $\Gamma$ to $\CC^\dotc$.

\begin{itenv}{Proposition}
  $\FF(\CC^\dotc,\Gamma)$ is a category for the composition law $(\Phi',\Phi)\mapsto\Phi'\botc\Phi$, where $(\Phi'\botc\Phi)(f)=\Phi'(f)\botc\Phi(f)$, if and only if $\Phi'(f)\botc\Phi(f)$ is defined for all $f\in\CC$.
\end{itenv}

\oldpage{3}

\begin{rmenv}{Definition}
  Let $\CC$ and $\CC_1$ be two double categories;
  we define a \emph{double functor} from $\CC$ to $\CC_1$ to be a map $\Phi$ from $\CC$ to $\CC_1$ such that $\Phi$ is a functor from $\CC^\dotc$ to $\CC_1^\dotc$ and a functor from $\CC^\botc$ to $\CC_1^\botc$.
  The class of double functors from $\CC$ to $\CC_1$ is denoted $\FF(\CC_1,\CC)$.
\end{rmenv}

\begin{itenv}{Proposition}
  $\FF(\CC_1,\CC)$ is a subcategory of $\FF(\CC_1^\dotc,\CC^\dotc)$ and of $\FF(\CC_1^\botc,\CC^\botc)$;
  endowed with the two induced composition laws, $\FF(\CC_1,\CC)$ is a double category.
\end{itenv}

\begin{itenv}{Proposition}
  \setcounter{footnote}{1}
  Let $\CC$ and $\CC'$ be two categories;
  the longitudinal category $\NN(\CC',\CC)$ of natural transformations~\footnote{\emph{Catégorie des foncteurs types}, Rev. Un. Mat. Argentina, \textbf{20}, 1960, p.~194.} between functors from $\CC$ to $\CC'$ can be identified with the category $\FF(\hsq\CC',\CC)$, by identifying the natural transformation $(\varphi',\tau,\phi)$ with the functor $\Phi$ such that
  \[
    \Phi(f)
    = \big(\varphi'(f),\,\tau(\beta(f)),\,\tau(\alpha(f)),\,\varphi(f)\big)
  \]
  for all $f\in\CC$.
\end{itenv}

Consequently, if $(\CC^\dotc,\CC^\botc)$ is a double category, then a functor $\Phi$ from a category $\Gamma$ into $\CC^\dotc$ can be considered as a generalised natural transformation from $\alpha^\botc\Phi$ to $\beta^\botc\Phi$.
We will see another generalisation of natural transformations (the double category of quintets) in a following publication.


\section{Structured categories}

Let $\MM_0$ be a class of classes such that if it contains $X$ then it also contains all the subsets of $X$, and if it contains $X$ and $X'$ then it also contains the product $X\times X'$;
let $\MM$ be the category of all functions from $X$ to $Y$, where $X,Y\in\MM_0$.
Let $(\MM,p,\KK,\SS)$ be a category of homomorphisms\footref{fn1}, with $\SS$ containing the groupoid of invertible elements of $\KK$;
let $\KK_0$ be the class of \todo of $\KK$;
we identify $h\in\KK$ with $(\beta^\KK(h),p(h),\alpha^\KK(h))$.

\begin{rmenv}{Definition}
  We define a \emph{structured category in $\KK$} to be a pair $(\CC^\dotc,s)$, where $\CC^\dotc$ is the structure of a category on $\CC\in\MM_0$, and $s\in\KK_0$ with $p(s)=\CC$, satisfying the following conditions:
  \begin{enumerate}
    \item There exists $s_0\in\KK_0$ such that \unsure{check the following}
      \[
        \begin{gathered}
          p(s_0)
          = \CC_0^\dotc
        \\(s,i_{\CC_0^\dotc},s_0),\,\,
          (s_0,\alpha,s),\,\,
          (s_0,\beta,s)
          \in\KK
        \end{gathered}
      \]
      where $i_{\CC_0^\dotc}$ is the canonical injection from $\CC_0^\dotc$ into $\CC$, and $\alpha$ and $\beta$ are the source and target maps (respectively) in $\CC^\dotc$.
    \item There exists a product $s\times s$ in $\KK$ such that $p(s\times s)=\CC\times\CC$;
      if $K$ is the subclass of $\CC\times\CC$ consisting of composible pairs, then there exists $s'\in\KK_0$ such that
      \[
        \begin{gathered}
          p(s')=K
        \\(s\times ,i_K,s')\in\KK.
        \end{gathered}
      \]
    \item writing $x$ to denote the map $(g,f)\mapsto g\dotc f$ from $K$ to $\CC$, the relation $(s\times s,i_K,s')\in\KK$ implies $(s,x,s')\in\KK$.
  \end{enumerate}
\end{rmenv}

\begin{rmenv}{Example}
  A structured category in $\tilde{\TT}$, where $\tilde{\TT}$ is the category of topologies, is a topological category.\footnote{\emph{Catégories topologiques et catégories différentiables}, Coll. Géo. Diff. Glo., Brussels, C.B.R.M., 1959, p.~137.}
\end{rmenv}

\oldpage{4}

\begin{itenv}{Theorem}
  For $(\CC^\dotc,\CC^\botc)$ to be a double category, it is necessary and sufficient that $(\CC^\dotc,\CC^\botc)$ be a structured category in the category $\FF$ of functors from one category to another;
  in this case, $(\CC^\botc,\CC^\dotc)$ is also a structured category in $\FF$ (the structure on $\CC^\dotc$ is $\CC^\botc$).
\end{itenv}

\end{document}
