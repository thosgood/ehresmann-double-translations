\documentclass{article}

\usepackage[margin=1.5in]{geometry}

\title{Double categories and structured categories}
\author{Charles Ehresmann}
\date{}

\newcommand{\origcit}{%
  \textsc{Ehresmann, C.}
  ``Catégories doubles et catégories structurées''.
  \emph{C. R. Acad. Sc.} \textbf{256} (1963), 1198--1201.
}


\usepackage{amssymb,amsmath}

\usepackage{hyperref}
\usepackage{xcolor}
\hypersetup{colorlinks,linkcolor={blue!50!black},citecolor={blue!50!black},urlcolor={blue!80!black}}
\usepackage{enumerate}

\usepackage{fouriernc}
\usepackage{Baskervaldx}




%% Theorem environments %%

\usepackage{amsthm}

\newenvironment{itenv}[1]
  {\phantomsection\par\medskip\noindent\textbf{#1.}\itshape}
  {\par\medskip}

\newenvironment{rmenv}[1]
  {\phantomsection\par\medskip\noindent\textbf{#1.}\rmfamily}
  {\par\medskip}


%% Shortcuts %%

\renewcommand{\geq}{\geqslant}
\renewcommand{\leq}{\leqslant}

\newcommand{\oldpage}[1]{\marginpar{\footnotesize$\Big\vert$ \textit{p.~#1}}}
\newcommand{\todo}{{\color{purple}\textbf{TO-DO }}}

\newcommand{\cd}{{\mathbin{\bullet}}}
\newcommand{\cb}{{\mathbin{\bot}}}


%% Document %%

\begin{document}

\maketitle
\origcit

\begin{abstract}
  Definition of structured categories; the particular case of double categories, which admit a \todo as a quotient category.
% \origcit
\end{abstract}

%% Content %%

\section{Double categories}
\oldpage{1}

\begin{rmenv}{Definition}
  We define a \emph{double category} to be a class $\mathcal{C}$ endowed with two composition laws, denoted $\cd$ and $\cb$, satisfying the following conditions:
  
  \begin{enumerate}
    \item $(\mathcal{C},\cd)$ is a category, denoted $\mathcal{C}^\cd$;
      the right and left \todo of $f\in\mathcal{C}$ will be denoted by $\alpha^\cd(f)$ and $\beta^\cd(f)$ respectively, and the class of \todo by $\mathcal{C}_0^\cd$;
    \item $(\mathcal{C},\cb)$ is a category, denoted $\mathcal{C}^\cb$;
      the \todo of $f\in\mathcal{C}^\cb$ will be denoted by $\alpha^\cb(f)$ and $\beta^\cb(f)$ respectively, and the class of \todo by $\mathcal{C}_0^\cb$;
    \item The maps $\alpha^\cd$ and $\beta^\cd$ (resp. $\alpha^\cb$ and $\beta^\cb$) are functors from $\mathcal{C}^\cb$ to $\mathcal{C}^\cb$ (resp. from $\mathcal{C}^\cd$ to $\mathcal{C}^\cd$);
    \item \emph{Axiom of permutability.}
      If the composites $k\cd h$, $g\cd f$, $k\cb g$, and $h\cb f$ are defined, then
      \[
        (k\cd h)\cb(g\cd f)
        = (k\cb g)\cd(h\cb f).
      \]
  \end{enumerate}

  Let $\mathcal{C}$ be a class endowed with two composition laws $\cd$ and $\cb$ satisfying axioms 1 and 2; consider the following axioms:

  \begin{enumerate}
    \item[3\textquotesingle.]
      $\mathcal{C}_0^\cd$ (resp. $\mathcal{C}_0^\cb$) is stable with respect to $\cb$ (resp. to $\cd$);
    \item[4\textquotesingle.]
      If the composites $k\cd h$, $g\cd f$, $k\cb g$, and $h\cb f$ are defined, then both $(k\cd h)\cb(g\cd f)$ and $(k\cb g)\cd(h\cb f)$ are defined and are equal to one another.
    \item[5.]
      For all $f\in\mathcal{C}$, we have
      \[
        \begin{aligned}
          \alpha^\cd(\alpha^\cb(f))
          = \alpha^\cb(\alpha^\cd(f)),
          &\qquad
          \beta^\cd(\beta^\cb(f))
          = \beta^\cb(\beta^\cd(f));
        \\\alpha^\cd(\beta^\cb(f))
          = \beta^\cb(\alpha^\cd(f)),
          &\qquad
          \alpha^\cb(\beta^\cd(f))
          = \beta^\cd(\alpha^\cb(f)).
        \end{aligned}
      \]
  \end{enumerate}
\end{rmenv}

\begin{itenv}{Proposition}
  For $(\mathcal{C},\cd,\cb)$ to be a double category, it is necessary and sufficient that conditions 1, 2, 3\textquotesingle, 4\textquotesingle, and 5 be satisfied.
  In this case, $\mathcal{C}_0^\cd$ (resp. $\mathcal{C}_0^\cb$) is a subcategory of $\mathcal{C}^\cd$ (resp. $\mathcal{C}^\cb$).
\end{itenv}

A \emph{double subcategory} of a double category $\mathcal{C}$ is a subclass $\mathcal{C}'$ of $\mathcal{C}$ that is a subcategory of $\mathcal{C}^\cd$ and of $\mathcal{C}^\cb$;
then $\mathcal{C}'$ is a double category for the composition laws induced by $\cd$ and $\cb$.

\begin{rmenv}{Definition}
  Let
\end{rmenv}

\end{document}