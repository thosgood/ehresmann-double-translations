\documentclass{article}

\usepackage[margin=1.6in]{geometry}

\title{Double categories and structured categories}
\author{Charles Ehresmann}
\date{}

\newcommand{\origcit}{%
  \textsc{Ehresmann, C.}
  ``Catégories doubles et catégories structurées''.
  \emph{C. R. Acad. Sc.} \textbf{256} (1963), 1198--1201.
}


\usepackage{amssymb,amsmath}

\usepackage{hyperref}
\usepackage{xcolor}
\hypersetup{colorlinks,linkcolor={blue!50!black},citecolor={blue!50!black},urlcolor={blue!80!black}}
\usepackage{enumerate}

\usepackage{mathrsfs}
\usepackage{fouriernc}




%% Theorem environments %%

\usepackage{amsthm}

\newenvironment{itenv}[1]
  {\phantomsection\par\medskip\noindent\textbf{#1.}\itshape}
  {\par\medskip}

\newenvironment{rmenv}[1]
  {\phantomsection\par\medskip\noindent\textbf{#1.}\rmfamily}
  {\par\medskip}


%% Shortcuts %%

\renewcommand{\geq}{\geqslant}
\renewcommand{\leq}{\leqslant}

\newcommand{\oldpage}[1]{\marginpar{\footnotesize$\Big\vert$ \textit{p.~#1}}}
\newcommand{\todo}{{\color{purple}\textbf{TO-DO }}}

\newcommand{\compdot}{{\mathbin{\cdot}}}
\newcommand{\compbot}{{\mathbin{\bot}}}


%% Document %%

\begin{document}

\maketitle
\origcit

\begin{abstract}
  Definition of structured categories; the particular case of double categories, which admit a \todo as a quotient category.
% \origcit
\end{abstract}

%% Content %%

\section{Double categories}
\oldpage{1}

\begin{rmenv}{Definition}
  We define a \emph{double category} to be a class $\mathcal{C}$ endowed with two composition laws, denoted $\compdot$ and $\compbot$, satisfying the following conditions:
  \begin{enumerate}
    \item $(\mathcal{C},\compdot)$ is a category, denoted by $\mathcal{C}^\compdot$;
      the right and left \todo
  \end{enumerate}
\end{rmenv}

\end{document}