\documentclass{article}

\usepackage[margin=1.6in]{geometry}

\title{Structured categories}
\author{Charles Ehresmann}
\date{1963}

\newcommand{\origcit}{%
  \textsc{Ehresmann, Charles.}
  ``Catégories structurées''.
  \emph{Ann. scient. Éc. Norm. Sup.} \textbf{80} (1963), 349--426.
  \url{http://www.numdam.org/item/?id=ASENS_1963_3_80_4_349_0}
}


%% Standards %%
\usepackage{amssymb}
\usepackage{amsmath}
\usepackage{hyperref}
\usepackage{xcolor}
\hypersetup{colorlinks,linkcolor={blue!50!black},citecolor={blue!50!black},urlcolor={blue!80!black}}
\usepackage{enumerate}
\usepackage{graphicx}
\usepackage{footmisc}


%% Typography %%

\usepackage{fouriernc}
\usepackage{Baskervaldx}

\renewcommand{\thesection}{\Roman{section}}


%% Theorem environments %%

\usepackage{amsthm}

\newenvironment{itenv}[1]
  {\phantomsection\par\medskip\noindent\textbf{#1.}\itshape}
  {\par\medskip}

\newenvironment{rmenv}[1]
  {\phantomsection\par\medskip\noindent\textbf{#1.}\rmfamily}
  {\par\medskip}


%% Shortcuts %%

\renewcommand{\geq}{\geqslant}
\renewcommand{\leq}{\leqslant}

\newcommand{\oldpage}[1]{\marginpar{\footnotesize$\Big\vert$ \textit{p.~#1}}}
\newcommand{\todo}{{\color{purple}\textbf{TO-DO }}}
\newcommand{\unsure}[1]{{\color{purple}\textbf{#1}}}

\newcommand{\CC}{\mathcal{C}}
\newcommand{\HH}{\mathcal{H}}
\renewcommand{\SS}{\mathcal{S}}
\newcommand{\MM}{\mathfrak{M}}
\usepackage{upgreek}
\newcommand{\comp}{\upchi}


%% Bibliography %%

\usepackage{biblatex}
\addbibresource{structured-categories.bib}
\renewbibmacro{in:}{%
  \ifboolexpr{%
     test {\ifentrytype{article}}%
  }{}{\printtext{\bibstring{in}\intitlepunct}}%
}


%% Document %%

\begin{document}

\maketitle

\hrule
\begin{itenv}{Note from the translator}
This document is a translation from French of the article

\medskip
{\normalfont\origcit}

\medskip
produced with permission from \todo
\end{itenv}
\hrule

\tableofcontents

%% Content %%

\section*{Introduction}
\addcontentsline{toc}{section}{\protect\numberline{}Introduction}

\oldpage{349}

This article is the first part of a work on the notion of structured categories and of \unsure{species of structured structures}.
The main results are summarised in a series in \emph{Notes à l'Académie des Sciences} \cite{3e}.

The first section begins with a short reminder on the \todo and categories of homomorphisms.
Let $(\CC,p,\HH,\SS)$ be a category of homomorphisms over $\CC$ such that $\SS$ contains the groupoid of invertible elements of the category $\HH$, and such that $\CC$ is further endowed with the structure of an inductive category.
We define the substructures of a structure of $\HH$.
This notion makes precise that of a sub-object of an arbitrary category, using the fact that $\HH$ is a category of homomorphisms and $\CC$ an inductive category;
it leads to endowing $\HH$ with the structure of an ordered category, which is the subject of the main results of this section.

Let $(\MM,p,\HH,\Gamma)$ be a category of homomorphisms with finite products, over a category $\MM$ of maps;
we define, at the start of Section II, $\HH$-structured categories (or, more precisely, $\HH(\HH',\HH'')$-structured categories).
We then give a certain number of examples: topological categories and differential categories \cite{3b}; double categories that arise, in particular, in the theory of natural transformations between functors \cite{3d}; order-structured categories, in particular inductive categories and inductive groupoids \cite{3c}, etc.
\oldpage{350}
These examples, which I was led to consider in the study of fibred spaces, foliated spaces, extensions of differential varieties, and local structures in general, are the origin of this work.
The end of Section II contains a series of general theorems:

\begin{itemize}
  \item $\HH$-structured functors form a category of homomorphisms over a category of functors, and over $\MM$;
    it has finite products and \unsure{resolutions to the right}.
  \item Let $(\CC^\bullet,s)$ be an $\HH$-structured category;
    if $\bar{\CC}^\bullet$ is a subcategory of $\CC^\bullet$, and $\bar{s}$ a substructure of $s$ such that $p(\bar{s})=\bar{C}$, then $(\bar{C}^\bullet,\bar{s})$ is an $\HH$-structured category.
  \item If $(\CC^\bullet,s)$ is an $\HH$-structured category, then the categories of \todo of $\CC^\bullet$ are $\HH$-structured categories.
\end{itemize}

All of these theorems use the additional hypothesis that $(\MM,p,\HH,\Gamma)$ is a category of homomorphisms \unsure{right resolved?} (that is, $\HH$ contains ``enough'' substructures).

The second part of this article (to appear soon) will contain the theory of \unsure{species of structured structures};
we will show how the \unsure{complete enlargement} procedure of an inductive groupoid can be generalised to \unsure{species of structured structures}.
We will then give some applications of all these concepts to more specific problems.



\section{Categories of homomorphisms and substructures}

\subsection{Conventions}

A category will be in general represented by the symbol denoted the \unsure{support class} of the category along with the symbol for the composition law that makes this class a category as a superscript.
For example: $\CC^\perp$, $\CC_1^\perp$, $\bar{\CC}^\perp$ (resp. $\CC^\bullet$, $\CC_1^\bullet$, $\bar{\CC}^\bullet$), \ldots denote the categories obtained by endowing the class $\CC$, $\CC_1$, $\bar{\CC}$, \ldots with the composition law $\perp$ (resp. $\bullet$).
The class of \unsure{identities} of a category will be denoted by the symbol representing the category along with a $0$ as a subscript.
For example: $\CC_0^\perp$, $(\CC_1^\perp)_0$, $\bar{\CC}_0^\perp$, \ldots.
If a class of objects is naturally associated to the category (for example, the classes in a category of maps from one class to another), then we will tacitly identify the \unsure{identities} with the corresponding objects.

Let $\CC^\perp$ be a category.
The source and target maps that send an element $f\in\CC^\perp$ to its right and left \unsure{identities} will be denoted $\alpha^\perp$ and $\beta^\perp$ (respectively).
The class of pairs $(g,f)$ such that the composite $g\perp f$ is defined (that is, such that $\alpha^\perp(g)=\beta^\perp(f)$) will be denoted by the symbol $\CC^\perp\star\CC^\perp$;
the map
\[
  (g,f) \mapsto g\perp f
  \qquad\text{where }(g,f\in\CC^\perp\star\CC^\perp)
\]
will be denoted by the symbol $\comp^\perp$.

\oldpage{351}
To simplify notation, if no confusion is possible, we will represent a category by the same symbol as its \unsure{support class};
in this case, it is to be understood that the composition law is denoted by $\bullet$;
we thus write $\CC$ instead of $\CC^\bullet$.
Similarly, we will also write $\CC_0$, $\alpha$, and $\beta$ instead of $\CC_0^\bullet$, $\alpha^\bullet$, and $\beta^\bullet$ (respectively).

Let $\bar{C}^\perp$ and $\CC^\perp$ be two categories;
the word ``functor'' will always mean a contravariant functor.
A functor from $\CC^\perp$ to $\bar{\CC}^\perp$ will be denoted either by a triple $(\bar{\CC}^\perp,F,\CC^\perp)$, where $F$ is the corresponding map, or by just the letter $F$.
The restriction of $F$ to the class $\CC_0^\perp$, considered as a map from $\CC_0^\perp$ to $\bar{\CC}_0^\perp$, will be denoted $F_0$.



\subsection{Reminder on \unsure{species of structures}}

Since the notion of \unsure{species of structures} \cite{3a} is essential in this article, we will recall the definition and main properties.

\begin{rmenv}{Definition 1}
  We say that a category $\CC$ is a \emph{category of operators on a class $\Sigma_0$} if we have defined a composition law $(f,z)\mapsto fz$ for certain pairs $(f,z)\in\CC\times\Sigma_0$ such that $fz\in\Sigma_0$ and such that the following axioms are satisfied:
  \begin{enumerate}
    \item \emph{Associativity.}
      If one of $g(fz)$ or $(g\bullet f)z$ is defined, then both of them are defined, and
      \[
        g(fz)
        = (g\bullet f)z;
      \]
    \item If $g\bullet f$ and $fz$ are defined, then $g(fz)$ is defined;
    \item Let $e\in\CC_0$;
      if $ez$ is defined, then $ez=z$;
    \item
      \begin{enumerate}
        \item For all $z\in\Sigma_0$, there exists at least one $f\in\CC$ such that $fz$ is defined;
        \item For all $f\in\CC$, there exists at least one $z\in\Sigma_0$ such that $fz$ is defined.
      \end{enumerate}
  \end{enumerate}
\end{rmenv}

These axioms imply that, for all $z\in\Sigma_0$, there exists exactly one $e\in\CC_0$ such that $ez$ is defined;
we thus obtain a map $p_0\colon z\mapsto e$ from $\Sigma_0$ to $\CC_0$;
we say that $z$ is a \emph{structure} on $p_0(z)$.

\begin{rmenv}{Definition 2}
  Let $\Sigma_0$ be a class, and $\CC$ a category;
  we say that $\Sigma_0$ is a \emph{\unsure{species of structures} over $\CC$} if we have a subcategory $\CC_1$ of $\CC$ that is a category of operators on $\Sigma_0$;
  let $p_0$ be the corresponding map from $\Sigma_0$ to $\CC_0$;
  we also say that $(\CC,p_0,\Sigma_0)$ is a \emph{\unsure{species of structures}}.
  If, further, $\CC_1=\CC$, then we say that $[\CC,p_0,\Sigma_0]$ is a \emph{\unsure{species of structures} on $\CC$}.
\end{rmenv}

Let $(\CC,p_0,\Sigma_0)$ be a \unsure{species of structures}.
Let $\Sigma$ be the class of pairs $(f,z)\in\CC\times\Sigma_0$ such that $fz$ is defined, i.e. such that $\alpha(f)=p_0(z)$.
Endowed with the composition law
\[
  (f',z')\bullet(f,z)
  = (f'\bullet f,z)
  \qquad\text{if and only if }z'=fz,
\]
\oldpage{352}
$\Sigma$ is a category, called the \emph{associated category of hypermorphisms} of the \unsure{species of structures} $(\CC,p_0,\Sigma_0)$.
The class of \unsure{identities} of $\Sigma$ can be identified with $\Sigma_0$ by associating $(e,z)$ with $z$.
The map $p_0$ extends to a functor $(\CC,p,\Sigma)$ satisfying the following property:
\begin{enumerate}
  \item[(E)] \itshape
    For all $h\in\Sigma$ and all $z\in\Sigma_0$ such that
    \[
      p_0(z)
      = p_0(\alpha(h))
    \]
    there exists exactly one $h'\in\Sigma$ such that
    \[
      p(h')=p(h)
      \qquad\text{and}\qquad
      \alpha(h')=z.
    \]
\end{enumerate}
The \unsure{species of structures} $(\CC,p_0,\Sigma_0)$ is also denoted by $(\CC,p,\Sigma)$.

Conversely, let $\CC$ and $\Sigma$ be two categories.
Let $(\CC,p,\Sigma)$ be a functor satisfying condition (E);
we say that $\Sigma$ is a \emph{category over $\CC$ with respect to $p$}.
We can show \cite{3a} that $p(\Sigma)$ is a subcategory of $\CC$, and that the map
\[
  h\mapsto (p(h),\alpha(h))
\]
lets us identify $\Sigma$ with the associated category of hypermorphisms of the \unsure{species of structures} $(\CC,p_0,\Sigma_0)$ in which the composition law is defined by
\[
  (f,z) \mapsto \beta(h)
\]
if and only if there exists $h\in\Sigma$ such that
\[
  f=p(h)
  \qquad\text{and}\qquad
  z=\alpha(h).
\]



\subsection{\unsure{Species of structures} \unsure{dominated} by a category}

We




%% Bibliography %%

\nocite{*}
\printbibliography[heading=bibintoc,title=Bibliography]

\end{document}
