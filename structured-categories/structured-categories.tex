\documentclass{article}

\usepackage[margin=1.45in]{geometry}

\title{Structured categories}
\author{Charles Ehresmann}
\date{1963}

\newcommand{\origcit}{%
  \textsc{Ehresmann, Charles.}
  ``Catégories structurées''.
  \emph{Ann. scient. Éc. Norm. Sup.} \textbf{80} (1963), 349--426.
  \url{http://www.numdam.org/item/?id=ASENS_1963_3_80_4_349_0}
}


%% Standards %%
\usepackage{amssymb}
\usepackage{amsmath}
\usepackage{hyperref}
\usepackage{xcolor}
\hypersetup{colorlinks,linkcolor={blue!50!black},citecolor={blue!50!black},urlcolor={blue!80!black}}
\usepackage{enumerate}
\usepackage{graphicx}
\usepackage{footmisc}


%% Typography %%

\usepackage{fouriernc}
\usepackage{Baskervaldx}

\renewcommand{\thesection}{\Roman{section}}


%% Theorem environments %%

\usepackage{amsthm}

\newenvironment{itenv}[1]
  {\phantomsection\par\medskip\noindent\textbf{#1.}\itshape}
  {\par\medskip}

\newenvironment{rmenv}[1]
  {\phantomsection\par\medskip\noindent\textbf{#1.}\rmfamily}
  {\par\medskip}


%% Shortcuts %%

\renewcommand{\geq}{\geqslant}
\renewcommand{\leq}{\leqslant}

\newcommand{\oldpage}[1]{\marginpar{\footnotesize$\Big\vert$ \textit{p.~#1}}}
\newcommand{\todo}{{\color{purple}\textbf{TO-DO }}}
\newcommand{\unsure}[1]{{\color{purple}\textbf{#1}}}

\newcommand{\CC}{\mathcal{C}}
\newcommand{\HH}{\mathcal{H}}
\renewcommand{\SS}{\mathcal{S}}
\newcommand{\MM}{\mathfrak{M}}


%% Bibliography %%

\usepackage{biblatex}
\addbibresource{structured-categories.bib}
\renewbibmacro{in:}{%
  \ifboolexpr{%
     test {\ifentrytype{article}}%
  }{}{\printtext{\bibstring{in}\intitlepunct}}%
}


%% Document %%

\begin{document}

\maketitle

\hrule
\begin{itenv}{Note from the translator}
This document is a translation from French of the article

\medskip
{\normalfont\origcit}

\medskip
produced with permission from \todo
\end{itenv}
\hrule

\tableofcontents

%% Content %%

\section*{Introduction}
\addcontentsline{toc}{section}{\protect\numberline{}Introduction}

\oldpage{349}

This article is the first part of a work on the idea of structured categories of of \unsure{species of structured structures}.
The main results are summarised in a series in \emph{Notes à l'Académie des Sciences} \cite{3e}\todo.

The first section begins with a short reminder on the \todo and categories of homomorphisms.
Let $(\CC,p,\HH,\SS)$ be a category of homomorphisms over $\CC$ such that $\SS$ contains the groupoid of invertible elements of the category $\HH$, and such that $\CC$ is further endowed with the structure of an inductive category.
We define the substructures of a structure of $\HH$.
This notion makes precise that of a sub-object of an arbitrary category, using the fact that $\HH$ is a category of homomorphisms and $\CC$ an inductive category;
it leads to endowing $\HH$ with the structure of an ordered category, which is the subject of the main results of this section.

Let $(\MM,p,\HH,\Gamma)$ be a category of homomorphisms with finite products, over a category $\MM$ of maps;
we define, at the start of Section II, $\HH$-structured categories (or, more precisely, $\HH(\HH',\HH'')$-structured categories).
We then give a certain number of examples: topological categories and differential categories \cite{3b}; double categories that arise, in particular, in the theory of natural transformations between functors \cite{3d}; order-structured categories, in particular inductive categories and inductive groupoids \cite{3c}, etc.

\todo


\section{Categories of homomorphisms and substructures}

\subsection{Conventions}

A






%% Bibliography %%

\nocite{*}
\printbibliography[heading=bibintoc,title=Bibliography]

\end{document}
