\documentclass{article}

\usepackage[margin=1.45in]{geometry}

\title{Structured categories}
\author{Charles Ehresmann}
\date{1963}

\newcommand{\origcit}{%
  \textsc{Ehresmann, Charles.}
  ``Catégories structurées''.
  \emph{Ann. scient. Éc. Norm. Sup.} \textbf{80} (1963), 349--426.
  \url{http://www.numdam.org/item/?id=ASENS_1963_3_80_4_349_0}
}


%% Standards %%
\usepackage{amssymb}
\usepackage{amsmath}
\usepackage{hyperref}
\usepackage{xcolor}
\hypersetup{colorlinks,linkcolor={blue!50!black},citecolor={blue!50!black},urlcolor={blue!80!black}}
\usepackage{enumerate}
\usepackage{graphicx}
\usepackage{footmisc}


%% Typography %%

\usepackage{fouriernc}
\usepackage{Baskervaldx}

\renewcommand{\thesection}{\Roman{section}}


%% Theorem environments %%

\usepackage{amsthm}

\newenvironment{itenv}[1]
  {\phantomsection\par\medskip\noindent\textbf{#1.}\itshape}
  {\par\medskip}

\newenvironment{rmenv}[1]
  {\phantomsection\par\medskip\noindent\textbf{#1.}\rmfamily}
  {\par\medskip}


%% Shortcuts %%

\renewcommand{\geq}{\geqslant}
\renewcommand{\leq}{\leqslant}

\newcommand{\oldpage}[1]{\marginpar{\footnotesize$\Big\vert$ \textit{p.~#1}}}
\newcommand{\todo}{{\color{purple}\textbf{TO-DO }}}
\newcommand{\unsure}[1]{{\color{purple}\textbf{#1}}}

\newcommand{\CC}{\mathcal{C}}
\newcommand{\HH}{\mathcal{H}}
\renewcommand{\SS}{\mathcal{S}}
\newcommand{\MM}{\mathfrak{M}}
\newcommand{\dotc}{{\mathbin{\bullet}}}
\newcommand{\botc}{{\mathbin{\bot}}}


%% Bibliography %%

\usepackage{biblatex}
\addbibresource{structured-categories.bib}
\renewbibmacro{in:}{%
  \ifboolexpr{%
     test {\ifentrytype{article}}%
  }{}{\printtext{\bibstring{in}\intitlepunct}}%
}


%% Document %%

\begin{document}

\maketitle

\hrule
\begin{itenv}{Note from the translator}
This document is a translation from French of the article

\medskip
{\normalfont\origcit}

\medskip
produced with permission from \todo
\end{itenv}
\hrule

\tableofcontents

%% Content %%

\unsure{fix $\dotc$ (and $\dotc$?)}

\section*{Introduction}
\addcontentsline{toc}{section}{\protect\numberline{}Introduction}

\oldpage{349}

This article is the first part of a work on the idea of structured categories of of \unsure{species of structured structures}.
The main results are summarised in a series in \emph{Notes à l'Académie des Sciences} \cite{3e}\todo.

The first section begins with a short reminder on the \todo and categories of homomorphisms.
Let $(\CC,p,\HH,\SS)$ be a category of homomorphisms over $\CC$ such that $\SS$ contains the groupoid of invertible elements of the category $\HH$, and such that $\CC$ is further endowed with the structure of an inductive category.
We define the substructures of a structure of $\HH$.
This notion makes precise that of a sub-object of an arbitrary category, using the fact that $\HH$ is a category of homomorphisms and $\CC$ an inductive category;
it leads to endowing $\HH$ with the structure of an ordered category, which is the subject of the main results of this section.

Let $(\MM,p,\HH,\Gamma)$ be a category of homomorphisms with finite products, over a category $\MM$ of maps;
we define, at the start of Section II, $\HH$-structured categories (or, more precisely, $\HH(\HH',\HH'')$-structured categories).
We then give a certain number of examples: topological categories and differential categories \cite{3b}; double categories that arise, in particular, in the theory of natural transformations between functors \cite{3d}; order-structured categories, in particular inductive categories and inductive groupoids \cite{3c}, etc.
\oldpage{350}
These examples, which I was led to consider in the study of fibred spaces, foliated spaces, extensions of differential varieties, and local structures in general, are the origin of this work.
The end of Section II contains a series of general theorems:

\begin{itemize}
  \item \todo
\end{itemize}

All of these theorems use the additional hypothesis that $(\MM,p,\HH,\Gamma)$ is a category of homomorphisms \unsure{right resolved?} (that is, $\HH$ contains ``enough'' substructures).

The second part of this article (to appear soon) will contain the theory of \unsure{species of structured structures};
we will show how the \unsure{complete enlargement} procedure of an inductive groupoid can be generalised to \unsure{species of structured structures}.
We will then give some applications of all these concepts to more specific problems.



\section{Categories of homomorphisms and substructures}

\subsection{Conventions}

A category will be in general represented by the symbol denoted the support class of the category along with the symbol for the composition law that makes this class a category as a superscript.
For example: $\CC^\botc$, $\CC_1^\botc$, $\bar{\CC}^\botc$ (resp. $\CC^\dotc$, $\CC_1^\dotc$, $\bar{\CC}^\dotc$), \ldots denote the categories obtained by endowing the class $\CC$, $\CC_1$, $\bar{\CC}$, \ldots with the composition law $\botc$ (resp. $\dotc$).






%% Bibliography %%

\nocite{*}
\printbibliography[heading=bibintoc,title=Bibliography]

\end{document}
